\documentclass[dvipdfmx,autodetect-engine]{jsarticle}
\usepackage{tikz}

\title{課題7}
\author{t180073}

\begin{document}

\maketitle

\section{区分求積法}
$\int_{0}^{1}x^{2}dx$を区分求積法で求めた場合次の式となる
\begin{equation}
    \frac{1}{n}\times\frac{1}{n^2}+\frac{1}{n}\times\frac{2^2}{n^2}+...+\frac{1}{n}\times\frac{n^2}{n^2}
\end{equation}
この式より
\begin{equation}
    \frac{1}{n^3}\times\frac{n}{6}\times(n+1)\times(2n+1)=\frac{1}{3}+\frac{1}{2n}+\frac{1}{6n^2}
\end{equation}
なので$n$を無限に飛ばすと$\frac{1}{3}$に収束。
\section{台形公式}
$\int_{0}^{1}x^{2}dx$を台形公式で求めた場合次の式となる
\begin{equation}
    \frac{1}{2n}\times(\frac{1}{n^2}+\frac{2}{n}^2)+\frac{1}{2n}\times(\frac{2}{n}^2+\frac{3}{n}^2)+...+\frac{1}{2n}\times(\frac{n-1}{n}^2+\frac{n}{n}^2)
\end{equation}
この式より
\begin{equation}
    \frac{1}{2n^3}\times((n\times(n-1)\times\frac{2n-1}{6}+n\times(n+1)\times\frac{2n+1}{6}-1^2)=\frac{1}{3}+\frac{1}{6n^2}
\end{equation}
なので$n$を無限に飛ばすと$\frac{1}{3}$に収束。
\section{区分求積法の誤差と台形公式の誤差どちらが小さいか}
区分求積法の求めた式は
\begin{equation}
    \frac{1}{3}+\frac{1}{2n}+\frac{1}{6n^2}
\end{equation}
となり台形公式の求めた式は
\begin{equation}
    \frac{1}{3}+\frac{1}{6n^2}
\end{equation}
なので$\frac{1}{2n}$の差より台形公式の誤差のほうが小さい。
\end{document}
